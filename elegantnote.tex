\documentclass[blue,normal,cn]{elegantnote}
% \usepackage{titlesec}
% \usepackage{titletoc}
\title{高等代数笔记}
\author{陈群}
% Elegant\LaTeX{} Group	\thanks{Elegant\LaTeX{} 其他模板下载地址:\href{https://ddswhu.me/resource/}{https://ddswhu.me/resource/}} }
\date{\today}
\usepackage{listings} 
\lstset{language=[LaTeX]{TeX},basicstyle=\footnotesize\ttfamily}

 
\begin{document}

\tableofcontents
{\color{ecolor}{\maketitle}}


% logo
% \centerline{∈cludegraphics[width=0.25\textwidth]{ElegantLaTeX_green.pdf}}
\section{行列式}
\subsection{n元排列}

\begin{definition}[$n$元排列]
$1,2,3,...,n$(或者$n$个不同的正整数)的全排列称为一个$n$元排列.
\end{definition}
显然,$1,2,3,...,n$(或者$n$个不同的正整数)的$n$元排列有$n!$个.

\begin{definition}[逆序数对和逆序数]
给定一个数对$(a,b),a ≠b $,若有a>b,则称$(a,b)$为逆序数对.一组数$1,2,3,...,n$中的逆序数对的个数称为该数列的逆序数,记作$τ(1,2,3,...,n)$
\end{definition}
对于数列$2431$,其中的逆序数对为$(2,1),(4,3),(4,1),(3,1)$,其逆序数为4.

\begin{definition}[奇排列与偶排列]
逆序数为奇数的排列称为奇排列;逆序数为偶数的排列称为偶排列.
\end{definition}

\begin{theorem}
对换排列中的两个数的位置会改变排列的奇偶性.
\end{theorem}

\begin{proof}
1.对换相邻位置的两个数.\\
设对换之前的排列为$a_1a_2...a_ia_j...a_n$,则对换之后的排列为:$a_1a_2...a_ja_i...a_n$,两个排列进行比计较可以知道,
对于前$i-1$个位置而言,互换前后的逆序数不变,对于第$j+1$到$n$个位置而言,互换前后的逆序数也没有改变.交换前,第$i,j$位置的逆序数和为:
\begin{equation*}
τ(a_i,a_j)+\sum_{k=j+1}^{n}\tau (a_i,a_k)+\sum_{k=j+1}^{n} \tau (a_j,a_k)
\end{equation*}
交换后,第$i,j$位置的逆序数和为:
\begin{equation*}
τ(a_j,a_i)+\sum_{k=j+1}^{n}\tau (a_j,a_k)+\sum_{k=j+1}^{n} \tau (a_i,a_k)
\end{equation*}

上述两式的差为$τ(a_j,a_i)-τ(a_i,a_j)=±1$,故两个排列的奇偶性相反.
\\
2.一般情况下\\
设对换之前的排列为$a_1a_2...a_ia_{i+1}...a_{i+s}a_j...a_n$,则对换之后的排列为:$a_1a_2...a_ja_{i+1}...a_{i+s}a_i...a_n$,
可以看做$s+1+s$次相邻元素之间的对换,每一次都改变了排列的奇偶性,由于$2s+1$为奇数,从而交换前后排列的奇偶性相反.
\end{proof}

\begin{theorem}
任一$n$元排列$j_1j_2...j_n$与$123...n$可以经过一系列的对换互变且所做的对换的次数与原排列$j_1j_2...j_n$的奇偶性相同.
\end{theorem}
\begin{proof}
设$j_1j_2...j_n$ 经过s次对换变为$123,...n$,显然$123,...n$是偶排列.由于每互换一次改变排列的奇偶性,所以
若$j_1j_2...j_n$为奇排列,则$s$为奇数;若$j_1j_2...j_n$为偶排列,则$s$为偶数
\end{proof}


\subsection{n阶行列式}

\begin{definition}[$n$阶行列式]
$n$阶行列式为$n!$项代数和,其中每一项是不同行不同列的$n$个元素的乘积.每一项按照行指标的自然顺序排列,列指标所成的排列是奇排列时带负号,偶排列时带正号.
\begin{equation*}
\begin{vmatrix} 
    a_{11}&a_{12} & a_{13} &...& a_{1n}\\
    a_{21}&a_{22} & a_{23} &...& a_{2n}\\
    ...&...&...&...&...\\
    a_{n1}&a_{n2} & a_{n3} &...& a_{nn}
\end{vmatrix}
:=\sum_{j_1j_2...j_n}(-1)^{\tau(j_1j_2...j_n)}a_{1j_1}a_{2j_2}...a_{nj_n}
\end{equation*}
记作$|A|$或者$\mathrm {det}A$
\end{definition}

特别地,可以写作:
\begin{equation*}
|A|:=\sum_{i_1i_2...i_n}(-1)^{\tau(i_1 i_2...i_n)}a_{i_1 1}a_{i_2 2}...a_{i_n n}
\end{equation*}

由上式可以得出结论:
\begin{property}
\begin{equation*}
    |A|:=|A^T|
\end{equation*}
\end{property}

\begin{property}
$A$作行变换,第$i$行乘以$k$,得到$A$
    \begin{equation*}
        A\stackrel{\textcircled{i}*k}{\longrightarrow}B
    \end{equation*}
得到
    \begin{equation*}
        |B|=k|A|
    \end{equation*}
\end{property}

\begin{proof}
    \begin{equation*}
        \begin{vmatrix} 
            a_{11}&a_{12} & a_{13} &...& a_{1n}\\
            ...&...&...&...&...\\
            a_{i1}&a_{i2} & a_{i3} &...& a_{in}\\
            ...&...&...&...&...\\
            a_{n1}&a_{n2} & a_{n3} &...& a_{nn}
        \end{vmatrix}
        \longrightarrow
        \begin{vmatrix} 
            a_{11}&a_{12} & a_{13} &...& a_{1n}\\
            ...&...&...&...&...\\
            ka_{i1}& ka_{i2} & ka_{i3} &...& ka_{in}\\
            ...&...&...&...&...\\
            a_{n1}&a_{n2} & a_{n3} &...& a_{nn}
        \end{vmatrix}
    \end{equation*}
$B$的行列式可以表示为:
\begin{equation*}
    \begin{aligned}
        |B|&=\sum_{j_1j_2...j_n}(-1)^{\tau(j_1 j_2...j_n)}a_{1 j_1}a_{2 j_2}...ka_{i,j_i}...a_{n j_n}\\
           &=k\sum_{j_1j_2...j_n}(-1)^{\tau(j_1 j_2...j_n)}a_{1 j_1}a_{2 j_2}...a_{i,j_i}...a_{n j_n}\\
           &=k|A|
    \end{aligned}
\end{equation*}
\end{proof}

\begin{property}
    \begin{equation*}
        \begin{aligned}
            &
            \begin{vmatrix} 
                a_{11}&a_{12} & a_{13} &...& a_{1n}\\
                ...&...&...&...&...\\
                b_1+c_1&b_2+c_2 & b_3+c_3 &...& b_n+c_n\\
                ...&...&...&...&...\\
                a_{n1}&a_{n2} & a_{n3} &...& a_{nn}
            \end{vmatrix}
        \\=&
        \begin{vmatrix} 
            a_{11}&a_{12} & a_{13} &...& a_{1n}\\
            ...&...&...&...&...\\
            b_1&b_2 & b_3 &...& b_n\\
            ...&...&...&...&...\\
            a_{n1}&a_{n2} & a_{n3} &...& a_{nn}\
            \end{vmatrix}
            +
            \begin{vmatrix} 
                a_{11}&a_{12} & a_{13} &...& a_{1n}\\
                ...&...&...&...&...\\
                c_1&c_2 & c_3 &...& c_n\\
                ...&...&...&...&...\\
                a_{n1}&a_{n2} & a_{n3} &...& a_{nn}
            \end{vmatrix}
        \end{aligned}
    \end{equation*}
    \end{property}
\begin{proof}

    设左式为$|A|$,则有
    \begin{equation*}
        \begin{aligned}
            |A|&=\sum_{j_1j_2...j_n}(-1)^{\tau(j_1 j_2...j_n)}a_{1 j_1}a_{2 j_2}...(b_{j_i}+c_{j_i})...a_{n j_n}\\
               &=\sum_{j_1j_2...j_n}(-1)^{\tau(j_1 j_2...j_n)}a_{1 j_1}a_{2 j_2}...b_{j_i}...a_{n j_n}+
                 \sum_{j_1j_2...j_n}(-1)^{\tau(j_1 j_2...j_n)}a_{1 j_1}a_{2 j_2}...c_{j_i}...a_{n j_n}\\
               &=
               \begin{vmatrix} 
                a_{11}&a_{12} & a_{13} &...& a_{1n}\\
                ...&...&...&...&...\\
                b_1&b_2 & b_3 &...& b_n\\
                ...&...&...&...&...\\
                a_{n1}&a_{n2} & a_{n3} &...& a_{nn}\
                \end{vmatrix}
                +
                \begin{vmatrix} 
                    a_{11}&a_{12} & a_{13} &...& a_{1n}\\
                    ...&...&...&...&...\\
                    c_1&c_2 & c_3 &...& c_n\\
                    ...&...&...&...&...\\
                    a_{n1}&a_{n2} & a_{n3} &...& a_{nn}
                \end{vmatrix}
        \end{aligned}
    \end{equation*}
\end{proof}

\begin{property}
    \begin{equation*}
        A\stackrel{(i,k)}{\longrightarrow}C \Rightarrow |C|=-|A|
    \end{equation*}
\end{property}


\begin{proof}
    设:
    \begin{equation*}
        |C|=   \begin{vmatrix} 
            a_{11}&a_{12} & a_{13} &...& a_{1n}\\
            ...&...&...&...&...\\
            a_{k1}&a_{k2} & a_{k3} &...& a_{kn}\\
            ...&...&...&...&...\\
            a_{i1}&a_{i2} & a_{i3} &...& a_{in}\\
            ...&...&...&...&...\\
            a_{n1}&a_{n2} & a_{n3} &...& a_{nn}
            \end{vmatrix}
            ,
        |A|=\begin{vmatrix} 
            a_{11}&a_{12} & a_{13} &...& a_{1n}\\
            ...&...&...&...&...\\
            a_{i1}&a_{i2} & a_{i3} &...& a_{in}\\
            ...&...&...&...&...\\
            a_{k1}&a_{k2} & a_{k3} &...& a_{kn}\\
            ...&...&...&...&...\\
            a_{n1}&a_{n2} & a_{n3} &...& a_{nn}
            \end{vmatrix}
    \end{equation*}
则有
    \begin{equation*}
        \begin{aligned}
            |C|&=  \sum_{j_1j_2...j_i...j_k...j_n}(-1)^{\tau(j_1 j_2...j_i...j_k...j_n)}a_{1 j_1}a_{2 j_2}...a_{k j_i}...a_{i j_k}...a_{n j_n}\\
               &= \sum_{j_1j_2...j_k...j_i...j_n}(-1)^{\tau(j_1 j_2...j_k...j_i...j_n)}a_{1 j_1}a_{2 j_2}...a_{i j_k}...a_{k j_i}...a_{n j_n}\\
                &=(-1)|A|=-|A|\\
            \end{aligned}
    \end{equation*}
\end{proof}


\begin{property}
    两行或者两列相等的行列式的值为0.
\end{property}

\begin{proof}
设$A$的两行相等,则互换相等的两行,行列式$|A|$不变,由上一性质,即$|A|=-|A|$,则$|A|=0$.
\end{proof}

\begin{property}
    行列式两行或者两列成比例,则该行列式为0.
\end{property}

\begin{proof}
    \begin{equation*}
        \begin{vmatrix} 
            a_{11}&a_{12} & a_{13} &...& a_{1n}\\
            ...&...&...&...&...\\
            a_{i1}&a_{i2} & a_{i3} &...& a_{in}\\
            ...&...&...&...&...\\
            la_{i1}&la_{i2} & la_{i3} &...& la_{in}\\
            ...&...&...&...&...\\
            a_{n1}&a_{n2} & a_{n3} &...& a_{nn}
            \end{vmatrix}
        =l0=0
        \end{equation*}
\end{proof}

\begin{property}
   \begin{equation*}
    A\stackrel{\textcircled{k}+\textcircled{i}l}{\longrightarrow}D \Rightarrow |A|=|D|
    \end{equation*}
\end{property}


\begin{proof}
    设
    \begin{equation*}
        |D|=
        \begin{vmatrix} 
            a_{11}&a_{12} & a_{13} &...& a_{1n}\\
            ...&...&...&...&...\\
            a_{i1}&a_{i2} & a_{i3} &...& a_{in}\\
            ...&...&...&...&...\\
            a_{k1}+la_{i1} & a_{k2}+la_{i2} & a_{k3}+la_{i3} &...& a_{kn}+la_{in}\\
            ...&...&...&...&...\\
            a_{n1}&a_{n2} & a_{n3} &...& a_{nn}
            \end{vmatrix}
        ,|A|=
            \begin{vmatrix} 
                a_{11}&a_{12} & a_{13} &...& a_{1n}\\
                ...&...&...&...&...\\
                a_{i1}&a_{i2} & a_{i3} &...& a_{in}\\
                ...&...&...&...&...\\
                a_{k1}&a_{k2} & a_{k3} &...& a_{kn}\\
                ...&...&...&...&...\\
                a_{n1}&a_{n2} & a_{n3} &...& a_{nn}
                \end{vmatrix}
        \end{equation*}
则有
    \begin{equation*}
        \begin{aligned}
        |D|&=
            \begin{vmatrix} 
                a_{11}&a_{12} & a_{13} &...& a_{1n}\\
                ...&...&...&...&...\\
                a_{i1}&a_{i2} & a_{i3} &...& a_{in}\\
                ...&...&...&...&...\\
                a_{k1} & a_{k2} & a_{k3} &...& a_{kn}\\
                ...&...&...&...&...\\
                a_{n1}&a_{n2} & a_{n3} &...& a_{nn}
            \end{vmatrix}
        +l
            \begin{vmatrix} 
                a_{11}&a_{12} & a_{13} &...& a_{1n}\\
                ...&...&...&...&...\\
                a_{i1}&a_{i2} & a_{i3} &...& a_{in}\\
                ...&...&...&...&...\\
                a_{i1}&a_{i2} & a_{i3} &...& a_{in}\\
                ...&...&...&...&...\\
                a_{n1}&a_{n2} & a_{n3} &...& a_{nn}
                \end{vmatrix}
                \\
        &=|A|+l0=|A|
        \end{aligned}
        \end{equation*}
\end{proof}

\section{线性空间}
\subsection{线性空间的定义}

\begin{definition}
    $$S\times M:=\{(a,b)|A∈ S,b ∈ M\}$$
称为$S$与$M$的笛卡尔积
\end{definition}

\begin{definition}
   非空集合$S$上的一个代数运算定义为$S\times S$到 $S$的一个映射.
\end{definition}

\begin{definition}
设$V$是一个非空集合,$K$是一个数域.若$V$上有一个运算,称为加法,即$(α,β) \mapsto α + β$;$K$与$V$之间有
运算,称为数量乘法,即$k\times V \rightarrow V:(k,α) \mapsto kα$,并且满足如下运算法则:
    \begin{enumerate}[(1)]
        \item $α+β=β+α,\quad \forall α , β ∈ V $
        \item $α+β+γ=α+(β+γ),\quad \forall α,β ∈ V $
        \item $V$中有元素$0$,使得:$α + 0=α,\quad \forall α ∈ V$
        \item $\forall α ∈ V$,有$β ∈ V$,满足$α+β=0$.称$β$为$α$的负元
        \item $1α=α,\quad \forall α ∈ V$
        \item $(kl)α=k(lα),\quad \forall k,l ∈ K,α ∈ V$
        \item $(k+l)α=kα+lα,\quad \forall k,l ∈ K,α ∈ V$
        \item $k(α+β)=kα+kβ,\quad \forall k,l ∈ K,α ∈ V$
    \end{enumerate}
    那么称$V$为数域$K$上的一个线性空间.$V$中元素可以称为向量,$V$可以称作向量空间.
\end{definition}
\begin{example}
    $\mathbb{R}^X:=$\{非空集合$X$到$R$的映射\},称为$X$上的实值函数.规定:
    \begin{equation*}
        \begin{aligned}
        &(f+g)(x):=f(x)+g(x),&\forall x ∈ X\\
        &(kf)(x):=kf(x), &\forall x ∈ X,k∈ \mathbb{R}\\
        \end{aligned}
    \end{equation*}\
    零函数为:
    \begin{equation*}
        0(x):=0,\forall x ∈ X
    \end{equation*}
    易证$\mathbb{R}^X$是$\mathbb{R}$上的一个线性空间.
\end{example}

设$V$是数域$K$上的线性空间,其具有以下性质:

\begin{property}
    $V$中的零元是唯一的.
\end{property}

\begin{proof}
    假设零元不唯一,设$0_1,0_2$都是$V$的零元,且有$0_1 ≠ 0_2 $.由$0_2$是零元,所以$0_1+0_2=0_1$;
    由$0_1$是零元,所以$0_2+0_1=0_2$,所以$0_1=0_2$,与假设矛盾,故假设不成立.
\end{proof}

\begin{property}
    对于任意的$α ∈ V$,$α$的负元是唯一的,记作$-α$.
\end{property}

\begin{proof}
    假设负元不唯一,设$β_1,β_2$都是$α$的负元,且有$β_1 ≠ β_2 $.则有
    \begin{equation*}
        \begin{aligned}
            β_1+α+β_2=β_1+(α+β_2)=β_1+0=β_1\\
            β_1+α+β_2= (β_1+α)+β_2=0+β_2=β_2\\
        \end{aligned}
    \end{equation*}
    所以$β_1=β_2$,与假设矛盾,故而假设不成立.
\end{proof}


\begin{property}
    $0α=0,\quad \forall α ∈ V$
\end{property}

\begin{proof}
    \begin{equation*}
        \begin{aligned}
        &0α=(0+0)α=0α+0α\\
        &0α+(-0α)=0α+0α+(-α)\\
        &0=0α
        \end{aligned}
    \end{equation*}
\end{proof}


\begin{property}
    $k0=0,\quad \forall k ∈ K$
\end{property}

\begin{proof}
    \begin{equation*}
        \begin{aligned}
        &k0=k(0+0)=k0+k0\\
        &k0+(-k0)=k0+k0+(-k0)\\
        &0=k0
        \end{aligned}
    \end{equation*}
\end{proof}

\begin{property}
   若$kα=0$,则有$k=0$或$α=0$
\end{property}

\begin{proof}
    假设$k≠ 0$
    \begin{equation*}
        α=1α=(k^{-1}k)α=k^{-1}(kα)
    \end{equation*}
    由$kα=0$,则有$α=0$
\end{proof}


\begin{property}
    $(-1)α=-α,\quad \forall α ∈ V$
 \end{property}
 
 \begin{proof}
     \begin{equation*}
         α+(-1)α=1α+(-1)α=(1-1)α=0
     \end{equation*}
     所以$(-1)α$是$α$的负元,为$-α$.
 \end{proof}

\subsection{线性子空间}
\begin{definition}
$V$是数域$K$上的线性空间,其中的元素为$α$,$U$是$V$的的一个非空子集.若$U$对于$V$的加法和数量乘法(以$V$中
的加法和数量乘法对$U$中的元素作运算)也是$K$上的线性空间,
则称$U$是$V$的子空间.
\end{definition}

\begin{theorem}
$V$的非空子集$U$是子空间 \\
$\Longleftrightarrow$
    \begin{enumerate}
        \item 若$α,β ∈ U$,则$α+β ∈ U$.($U$对于$V$的加法封闭)
        \item 若$α ∈ U$,则$kα ∈ U$.($U$对于$V$的数量乘法封闭)
    \end{enumerate}
\end{theorem}

\begin{proof}
$"\Rightarrow"$:有定义可以得到;\\
$"\Leftarrow"$:$V$的加法和数量乘法限定到$U$上为$U$的加法和数量乘法,显然8条运算法则中,1,2,5,6,7,8是成立的.对于第3和第4条
\\3.由$U≠ \varnothing,\exists β ∈ U$,使得,$0β=0 ∈ U$,从而$0 ∈ U$,零元存在.\\
4.对于任意的$α ∈ U$,有$(-1)α ∈ U$,即$-α ∈ U$,从而负元存在.
\end{proof}

\begin{example}
$\{0\}$空间是子空间
\end{example}

\begin{definition}
若存在一组向量$α_1,α_2,...,α_s ∈ W$,则将
$k_1 α_1+k_2 α_2+...+k_s α_s$称为$α_1,α_2,...,α_s$的线性组合.其中$k_1,k_2,...,k_s ∈ K$
.令$W=\{k_1 α_1+k_2 α_2+...+k_s α_s|k_1,k_2,...,k_s ∈ k\}$,则有$0 ∈ W$且$W$对于$V$的加法和数量乘法
封闭,从而$W$是$V$的子空间,将其称作由向量组$α_1,α_2,...,α_s$生成的子空间,记作$<α_1,α_2,...,α_s>$.
\end{definition}

\begin{definition}
   $β ∈ <α_1,α_2,...,α_s>
   \\ :\Longleftrightarrow  \quad$
    存在$l_1,l_2,...,l_s ∈ K$
    使得$β=l_1 α_1+l_2 α_2+...+l_s α_s$.
    \\此时称,$β$可以由$α_1,α_2,...,α_s$线性表出.
\end{definition}

对于方程组:\\
$x_1α_1+x_2α_2+...+x_nα_n=β$ $
\\\Longleftrightarrow$
$K$中有一组$c_1,c_2,...,c_n$,使得$c_1α_1+c_2α_2+...+c_nα_n=β$
$\\\Longleftrightarrow$
$β$可以由$α_1,α_2,...,α_n$线性表出.
$\\\Longleftrightarrow$
$β ∈ <α_1,α_2,...,α_n>$(生成的子空间).

\subsection{线性相关与线性无关}
\begin{definition}
    设$V$是数域$K$上的一个线性空间,对于$V$中的一个向量组:$α_1,α_2,...,α_s,(s≥1)$,若$K$中有不全为$0$的一组数,使得:
    \begin{equation*}
        k_1α_1+k_2 α_2+...+k_s α_s=0
    \end{equation*}
    则称$α_1,α_2,...,α_s$线性相关,否则,称$α_1,α_2,...,α_s$线性无关.即:
    \begin{equation*}
        k_1 α_1+k_2 α_2+...+k_s α_s=0 \quad \Longrightarrow \quad k_1=k_2=...=k_s=0
    \end{equation*}
    则$α_1,α_2,...,α_s$线性无关.
\end{definition}

对于线性方程组

\begin{enumerate}[(1)]
    \item 若存在不全为$0$的$c_1,c_2,...c_n$,使得$c_1α_1+c_2α_2+...+c_nα_n=0 \quad \Longleftrightarrow \quad $方程组有非零解. 
    \item $K^s$中的列向量$α_1,α_2,...,α_n$线性无关.$\quad \Longleftrightarrow \quad $齐次线性方程组$x_1α_1+x_2α_2+...+x_nα_n$只有零解.
\end{enumerate}

$K^n$中的列向量$α_1,α_2,...,α_n$线性相关$\quad \Longleftrightarrow \quad $ 以$α_1,α_2,...,α_n$为列向量的矩阵$A$的行列式:$|A|=0$

$K^n$中的列向量$α_1,α_2,...,α_n$线性无关$\quad \Longleftrightarrow \quad $ 以$α_1,α_2,...,α_n$为列向量的矩阵$A$的行列式:$|A|≠ 0$

行向量的性质与上述情况相同.

\subsection{线性相关和线性无关的向量组}
\begin{enumerate}[(1)]
    \item $α$线性相关 $\quad \Longleftrightarrow \quad \exists k≠0$,使得 $kα=0 \quad \Longleftrightarrow \quad α=0$\\从而$α$线性无关  $\quad \Longleftrightarrow \quad  α ≠ 0$
    \item 对于向量组$α_1,α_2,...,α_n$中,如果有一部分向量组线性相关,则$α_1,α_2,...,α_n$ 线性相关.
    \item 向量组线性无关   $\quad \Longrightarrow \quad$ 不存在部分向量组线性相关(任何部分向量组线性无关)
    \item 向量组 $α_1,α_2,...,α_s$ 线性相关 $\quad \Longleftrightarrow \quad$由定义可得,存在$K$中不全为$0$的数$k_1,k_2,...,k_s$使得:
        \begin{equation*}
            k_1α_1+k_2α_2+...+k_sα_s=0
        \end{equation*}
        不妨设$k_i≠ 0$,则有
        \begin{equation*}
            α_i=-\frac{1}{k_i}(k_1α_1+k_2α_2+...+k_{i-1}α_{i-1}+k_{i+1}α_{i+1}+k_sα_s)
        \end{equation*}
        $\quad \Longleftrightarrow \quad$ $α_1,α_2,...,α_s$ 至少有一个向量可以由其余的向量线性表出.其中第一个条件的必要性("$\Leftarrow$")的证明为:
        \begin{proof}
            设$α_j=l_1α_1+...+l_{j-1} α_{j-1}+l_{j+1} α_{j+1}+...+l_s α_s$\\
            $\quad \Longrightarrow \quad 0=l_1α_1+...+l_{j-1} α_{j-1}-α_j+l_{j+1} α_{j+1}+...+l_s α_s   \quad \Longrightarrow \quad $
            $α_1,α_2,...,α_s$线性相关.
        \end{proof}
        则有逆否命题:
        \\向量组 $α_1,α_2,...,α_s$ 线性无关$\quad \Longleftrightarrow \quad$其中每一个向量都不能由其余的向量线性表出.
\end{enumerate}


\begin{proposition}
    \label{linear_show_only}
若$β$可以由向量$α_1,α_2,...,α_s$,其表出方式是唯一的充分必要条件是$α_1,α_2,...,α_s$线性无关.
\end{proposition}

\begin{proof}
    "$\Leftarrow$":\\
    设
    \begin{equation}
    β=a_1α_1+a_2α_2+...+a_sα_s  \label{beta_1}
    \end{equation}
    假设表出不唯一,即$β$还可以表示成:
    \begin{equation}
    β=b_1 α_1+b_2 α_2+...+b_s α_s  \label{beta_2}
    \end{equation}
    其中$a_i ≠ b_i$.
    \eqref{beta_2}-\eqref{beta_1}可以得到:
    \begin{equation*}
        0=(b_1-a_1)α_1+(b_2-a_2)α_2+...+(b_s-a_s)α_s 
    \end{equation*}
    由$α_1,α_2,...,α_s$线性无关,可以得到:
    \begin{equation*}
        a_i-b_i=0,\quad i=1,2,...,s
    \end{equation*}
    即$a_i=b_i$,假设不成立,$β$的表出方式是唯一的.\\
    "$\Rightarrow$":\\
    反证法.假设$α_1,α_2,...,α_s$线性相关,则有$K$中的$k_1,k_2,...,k_s$($k_i$不全为零)使得:
    \begin{equation*}
        0=k_1α_1+k_2α_2+...+k_sα_s 
    \end{equation*}
    此外,由
    \begin{equation}
    β=a_1α_1+a_2α_2+...+a_sα_s  \label{beta_3}
    \end{equation}
    可以得到:
    \begin{equation}
        β=(a_1+k_1)α_1+(a_2+k_2)α_2+...+(a_s+k_s)α_s  \label{beta_4}
    \end{equation}
    由$k_1,k_2,...,k_s$不全为零,不妨设$k_i ≠ 0$,则有$a_i+k_i ≠ a_i$.则\eqref{beta_3}式和\eqref{beta_4}式相比较,至少有一项是不相等的(第$i$个位置),从而$β$有两种不相同的表出方式,这与已证明的必要性矛盾,故而假设不成立,$α_1,α_2,...,α_s$是线性无关的.
\end{proof}

\begin{proposition}
    设$α_1,α_2,...,α_s$线性无关,若$α_1,α_2,...,α_s,β$线性相关,那么$β$可以由$α_1,α_2,...,α_s$线性表出.
\end{proposition}
\begin{proof}
    由$α_1,α_2,...,α_s,β$线性相关,则存在不全为零的一组数$k_1,k_2,...,.k_s,l$,使得:
    \begin{equation*}
        k_1 α_1+k_2 α_2+...+k_s α_s+l β =0 
    \end{equation*}
    若$l=0$,则有
    \begin{equation*}
        k_1 α_1+k_2 α_2+...+k_s α_s=0 
    \end{equation*}
    由$α_1,α_2,...,α_s$线性无关,则有$k_1,k_2,...,.k_s$全为零,再加上$l$,有$k_1,k_2,...,.k_s,l$全为零,与$α_1,α_2,...,α_s,β$线性相关矛盾,假设不成立.所以$l\neq 0$.
    则有
    \begin{equation*}
         β =-\frac{k_1}{l} α_1-\frac{k_2}{l} α_2-...-\frac{k_s}{l} α_s 
    \end{equation*}
\end{proof}


\subsection{极大线性无关组}
\begin{definition}
    \begin{equation*}
        <α_1,α_2,...,α_s>:=\{k_1 α_1+k_2 α_2+...+k_s α_s|k_i∈ K,i=1,2,...,s\}
    \end{equation*}
称为$α_1,α_2,...,α_s$生成的向量空间.
\end{definition}


\begin{definition}
    $α_1,α_2,...,α_s$的一个部分组若满足:
    \begin{enumerate}[(a)]
        \item 该部分组线性无关.
        \item 从$α_1,α_2,...,α_s$中其余的向量中(若有)任取一个向量添加进来得到的新的部分组都线性相关.
    \end{enumerate}
    则称它是为$α_1,α_2,...,α_s$的极大线性无关组.
\end{definition}

\begin{definition}
    若向量组$α_1,α_2,...,α_s$中的每一个向量都可以由$β_1,β_2,...,β_r$线性表出,则称$α_1,α_2,...,α_s$可以由$β_1,β_2,...,β_r$线性表出.
    若$α_1,α_2,...,α_s$和$β_1,β_2,...,β_r$可以相互线性表出,则称这两个向量组是等价的,记作:
    \begin{equation*}
        \{α_1,α_2,...,α_s\} \cong \{β_1,β_2,...,β_r\}
    \end{equation*}
\end{definition}
% congruent :全等  一致的的  等价的

\begin{proposition}
    $α_1,α_2,...,α_s$和它的任意一个极大线性无关组等价.
\end{proposition}

\begin{proof}
    不妨设$α_1,α_2,...,α_s$的一个极大线性无关组为$α_1,α_2,...,α_m(m≤s)$
    \begin{equation*}
        α_j=0α_1+...+α_j+...+0α_s,\quad j=1,2,...,m
    \end{equation*}

    故而$α_1,α_2,...,α_m$中的所有的元素都可以由$α_1,α_2,...,α_s$线性表出.

    同时显然$α_1,α_2,...,α_m$可以由$α_1,α_2,...,α_m$线性表出.对于$α_j(m<j≤s)$,将其添加到$α_1,α_2,...,α_m$中,形成$α_1,α_2,...,α_m,α_j$,由极大线性无关组的定义,$α_1,α_2,...,α_m,α_j$线性相关,由前一命题可得,$α_j$
    可以由$α_1,α_2,...,α_m$线性表出,从而$α_1,α_2,...,α_s$可以由$α_1,α_2,...,α_m$线性表出.

    故而,$α_1,α_2,...,α_s$和$α_1,α_2,...,α_m$可以相互线性表出,$α_1,α_2,...,α_s$和它的任意一个极大线性无关组等价.
\end{proof}

\begin{property}
    等价具有以下的性质:
    \begin{enumerate}
            \item 每一个向量与其自身等价(反身性)
            \item 若$ \{α_1,α_2,...,α_s\} \cong \{β_1,β_2,...,β_r\} \Rightarrow \{β_1,β_2,...,β_r\} \cong \{α_1,α_2,...,α_s\}$,对称性
            \item  $\{α_1,α_2,...,α_s\} \cong \{β_1,β_2,...,β_r\},\{β_1,β_2,...,β_r\} \cong \{γ_1,γ_1,...,γ_t\} \Rightarrow  \{α_1,α_2,...,α_s\} \cong \{γ_1,γ_1,...,γ_t\}$,传递性
    \end{enumerate}
\end{property}
\begin{proof}
第三个性质的证明。
    \begin{equation*}
        \begin{aligned}
            α_i=\sum_{j=1}^{r}a_{ij}β_{j},i=1,2,...,s\\
            β_j=\sum_{l=1}^{t}b_{jl}γ_{l},j=1,2,...,r\\
        \end{aligned}
    \end{equation*}

    则有
    \begin{equation*}
        \begin{aligned}
            α_i&=\sum_{j=1}^{r}a_{ij}\sum_{l=1}^{t}b_{jl}γ_{l}=\sum_{j=1}^{r}\sum_{l=1}^{t}a_{ij}b_{jl}γ_{l}\\
            &=\sum_{l=1}^{t}\sum_{j=1}^{r}a_{ij}b_{jl}γ_{l}=\sum_{l=1}^{t}γ_{l}\left[\sum_{j=1}^{r}a_{ij}b_{jl}γ_{l}\right]
        \end{aligned}
    \end{equation*}
    所以有$α_i$可以由$γ_1,γ_2,...,γ_t$线性表出。
\end{proof}
由以上等价关系的对称性和传递性可以得到:
\begin{proposition}
向量组$α_1,α_2,...,α_s$的任意两个极大线性无关组等价。
\end{proposition}

\begin{lemma}
若$β_1,β_2,...,β_r$可以由$α_1,α_2,...,α_s$线性表出,若$r>s$,那么$β_1,β_2,...,β_r$一定线性相关。
\end{lemma}

\begin{proof}
由已知可以得到
    \begin{equation*}
        \begin{aligned}
            &β_1=a_{11} α_1+...+  a_{s1}α_s\\
            &β_2=a_{12} α_1+...+  a_{s2}α_s\\
            &\vdots\\
            &β_r=a_{1r} α_1+...+  a_{sr}α_s\\
        \end{aligned}
    \end{equation*}
    令
    \begin{equation*}
        x_1β_1+x_2β_2+...+x_rβ_r=0
    \end{equation*}
    则有
    \begin{equation*}
        \begin{aligned}
            &x_1(a_{11} α_1+...+  a_{s1}α_s)\\
            +&x_2(a_{12} α_1+...+  a_{s2}α_s)\\
            &\vdots\\
            +&x_r(a_{1r} α_1+...+  a_{sr}α_s)=0\\
        \end{aligned}
    \end{equation*}
    \begin{equation*}
        \Rightarrow 
        (a_{11}x_1+...+a_{1r}x_r)α_1+...+(a_{s1}x_1+...+a_{sr}x_r)α_s=0
    \end{equation*}
    取向量组$\alpha_1,\alpha_2,...,\alpha_s$中的一个极大线性无关组,不妨设其为$\alpha_1,\alpha_2,...,\alpha_m$,且有$m≤s<r$,则有方程组:
    \begin{equation*}
        \begin{aligned}
            &a_{11} x_1+...+  a_{1r}x_r=0\\
            &a_{21} x_1+...+  a_{2r}x_r=0\\
            &\vdots\\
            &a_{m1} x_1+...+  a_{mr}x_r=0\\
        \end{aligned}
    \end{equation*}
    由$m≤s<r$,方程组由非零解,设非零解为$k_1,k_2,...,k_r$,得到
    \begin{equation*}
        k_1β_1+k_2β_2+...+k_rβ_r=0
    \end{equation*}
    $k_1,k_2,...,k_r$不全为零,从而$β_1,β_2,...,β_r$线性相关。
\end{proof}


\begin{lemma} 
    \label{1}
    若$β_1,β_2,...,β_r$可以由$α_1,α_2,...,α_s$线性表处,且$β_1,β_2,...,β_r$线性无关,那么$r≤s$.
\end{lemma}

\begin{corollary}
   等价的线性无关的两个向量组所含有的向量的个数相等。即
   \begin{equation*}
    \{α_1,α_2,...,α_m\} \cong \{γ_1,γ_2,...,γ_s\} \Rightarrow m=s
   \end{equation*}
\end{corollary}

\begin{proof}
    由前一推论可以得到。
\end{proof}


\begin{corollary}
    $α_1,α_2,...,α_m$的任意两个极大线性无关组所含有的向量个数相等。
 \end{corollary}

\begin{definition}
    向量组$α_1,α_2,...,α_s$的任意一个极大线性无关组所含有的向量个数称为$α_1,α_2,...,α_s$的秩(rank).只含有0的向量组的秩规定为0.
    记作 $\mathrm{rank}\{α_1,α_2,...,α_s\}$
\end{definition}

\begin{proposition}
向量组$α_1,α_2,...,α_s$线性无关 \\
$\Leftrightarrow α_1,α_2,...,α_s$是其本身的极大线性无关组\\
$\Leftrightarrow \mathrm{rank}\{α_1,α_2,...,α_s\}=s$
\end{proposition}

\begin{proposition}
    向量组(\uppercase\expandafter{\romannumeral1})可以由向量组(\uppercase\expandafter{\romannumeral2})线性表出,则$\mathrm{rank}(\uppercase\expandafter{\romannumeral1}) \leq \mathrm{rank}(\uppercase\expandafter{\romannumeral2})$
\end{proposition}

\begin{proof}
    取(\uppercase\expandafter{\romannumeral1})中的极大线性无关组(\uppercase\expandafter{\romannumeral1})',(\uppercase\expandafter{\romannumeral2})中极大线性无关组(\uppercase\expandafter{\romannumeral2})',

    \begin{equation*}
        \begin{aligned}
            (\uppercase\expandafter{\romannumeral1})\cong (\uppercase\expandafter{\romannumeral1})'\\
            (\uppercase\expandafter{\romannumeral2}) \cong (\uppercase\expandafter{\romannumeral2})'
        \end{aligned}
    \end{equation*}
    则有
    \begin{equation*}
        (\uppercase\expandafter{\romannumeral1})'\cong (\uppercase\expandafter{\romannumeral1})'\\
    \end{equation*}
    即 (\uppercase\expandafter{\romannumeral1})' 可以由  (\uppercase\expandafter{\romannumeral2})'线性表出.由引理\ref{1},  (\uppercase\expandafter{\romannumeral1})'所含向量的个数 $\leq$ (\uppercase\expandafter{\romannumeral2})'所含向量的个数。即为
    $\mathrm{rank}(\uppercase\expandafter{\romannumeral1}) \leq \mathrm{rank}(\uppercase\expandafter{\romannumeral2})$
\end{proof}

\begin{corollary}
    \label{4}
等价的向量组的秩相等。
\end{corollary}




 \subsection{基、维数和坐标}

\begin{definition}
    设$V$是数域$K$上任意的线性空间。\\
    $V$的一个有限子集$\{α_1,α_2,...,α_s\}$线性相(无)关
    $:\Leftrightarrow$向量组$α_1,α_2,...,α_s$线性相(无)关\\
    $V$的一个无限子集$S$线性相关
    $:\Leftrightarrow S$有一个有限子集是线性相关的\\
    从而有,$V$的一个无限子集$S$线性无关 $:\Leftrightarrow S$任何一个有限子集是线性无关的
\end{definition}

\begin{definition}
    设$V$是数域$K$上任意的线性空间。$V$的一个子集$S$若满足以下条件:
    \begin{enumerate}
        \item $S$是线性无关的
        \item $V$中的任意向量可以由$S$中的有限多个向量线性表出
    \end{enumerate}
    则称$S$是$V$的一个基.
\end{definition}

若$S=\{α_1,α_2,...,α_s\}$,则向量组$α_1,α_2,...,α_s$是$V$的一个(有序)基($∅$规定为线性无关)

\begin{proposition}
任何一个线性空间都有一个基。
\end{proposition}

\begin{definition}
    若$V$有一个基是有限维的,则称$V$是有限维的。\\
    若$V$有一个基是无限维的,则称$V$是无限维的。
\end{definition}

\begin{theorem}
    \label{thr_2}
    若$V$是有限维的,则$V$的任意两个基所含有向量的个数相同。
\end{theorem}

\begin{proof}
    设$V$有一个基$ α_1,α_2,...,α_n$,任取另外一个基$S$,假设$S$所含有的向量个数$k>n$,则$S$中可以取出$β_1,β_2,..,β_{n+1}$,且$β_1,β_2,..,β_{n+1}$可以由$ α_1,α_2,...,α_n$线性表出。由$n+1>n$,据引理\ref{1},$β_1,β_2,..,β_{n+1}$线性相关,这与$S$是一个基矛盾,假设不成立,进一步有$k<=n$
    \begin{equation*}
        \{β_1,β_2,..,β_{k}\} \cong \{α_1,α_2,...,α_n\}
    \end{equation*}
    由两个向量组都是线性无关的,从而由推论\ref{4},等价的线性无关向量组含有相同的个数,从而$k=n$
\end{proof}

\begin{corollary}
    若$V$是无限维的线性空间,则$V$的任意一个基都是无限子集。 
\end{corollary}

\begin{proof} 
    假设$V$有一个基是有限子集,由$V$中任何一个基的向量个数相同,这与$V$是无限维矛盾。
\end{proof}


\begin{definition}
    若$V$是有限维的,则将$V$的任何一个基所含有的向量个数称为$V$的维数,记作$\mathrm{dim}_K V$或者$\mathrm{dim} V$;若$V$是无限维的,则将$V$的维数记成$\mathrm{dim} V=∞$;$\{0\}$的维度规定为0.
\end{definition}

\begin{proposition}
    \label{12}
    设$\mathrm{dim} V=n$,则$V$任意$n+1$个向量都线性相关。
\end{proposition}

\begin{proof}
    设$α_1,α_2,...,α_n$是$V$的一个基,则对于$V$中的任意$n+1$个向量$β_1,β_2,...,β_n,β_{n+1}$都可以由$α_1,α_2,...,α_n$线性表出,由$n+1>n$,从而$β_1,β_2,...,β_n,β_{n+1}$线性相关。
\end{proof}

\begin{definition}
    设$\mathrm{dim} V=n$,取$V$的一个基$α_1,α_2,...,α_n$,则$V$中的任意一个向量可以表示成
    \begin{equation*}
        α=a_1 α_1+a_2  α_2+...+a_n  α_n
    \end{equation*}
    且表示方式唯一。将$(a_1,a_2,...,a_n)$称为$α$在基$α_1,α_2,...,α_n$下的坐标。
\end{definition}

\begin{proof}
    由定义\ref{linear_show_only}可得.
\end{proof}

对于
\begin{equation*}
     e_1=\begin{pmatrix} 
            1\\
            0\\
            0\\
            \vdots\\
            0
         \end{pmatrix}
         ,e_2=\begin{pmatrix} 
                0\\
                1\\
                0\\
                \vdots\\
                0
            \end{pmatrix}
            ,\dots
        ,e_n=\begin{pmatrix} 
            0\\
            0\\
            0\\
            \vdots\\
            1
         \end{pmatrix}
\end{equation*}
\begin{equation*}
    α=\begin{pmatrix} 
        a_1\\
        a_2\\
        a_3\\
        \vdots\\
        a_n
     \end{pmatrix}
     =a_1e_1+a_2e_2+...+a_3e_3
\end{equation*}

$e_1,e_2,...,e_n$为$K_n$的一个基(标准基)。

\begin{proposition}
    设$\mathrm{dim} V=n$,则$V$中任意$n$个线性无关的向量都是$V$的一个基。
\end{proposition}

\begin{proof}
    从$V$中取线性无关的向量组$α_1,α_2,...,α_n$,任取$β ∈ V$,由命题\ref{12},$α_1,α_2,...,α_n,β$线性相关,从而$α_1,α_2,...,α_n$是极大线性无关组,$β$可以由$α_1,α_2,...,α_n$线性表出。由$β$的任意性可以得出,$α_1,α_2,...,α_n$是$V$的一个基。
\end{proof}



\end{document}
